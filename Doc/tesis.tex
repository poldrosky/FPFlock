\documentclass[bsc,letterpaper,12pt]{csthesis}
%\documentclass[letter,12pt,twoside]{csthesis}  para imprimir de lado y lado


\paperwidth = 21.6cm		% Tamaño de la hoja
\paperheight = 27.9cm

%Margenes horizontales
\hoffset = -2.54cm		% Se elimna el offset horizontal
\oddsidemargin = 4cm
\evensidemargin = 2cm
\textwidth = 15.6cm

%Margenes verticales
\voffset = 0.46cm
\topmargin = 0cm
\headheight = 0cm
\headsep = 0cm
\textheight = 21.9cm
\footskip = 1.2cm



\usepackage[spanish]{babel}     % Idioma Capitulos y demas
\usepackage[utf8]{inputenc}
\usepackage{lmodern}
\usepackage[T1]{fontenc}
\usepackage{cc/CreativeCommons}  %Licencia

\usepackage{listings} % Ingresar Codigo Fuente
\usepackage{verbatim}
\usepackage{moreverb}
\let\verbatiminput=\verbatimtabinput
\def\verbatimtabsize{8}  % Tabulación en verbatim

% Paquete para el manejo de hipervinculos
\usepackage[b5paper, breaklinks=true, pdfborder={0 0 0},colorlinks=false,pageanchor=true, 
plainpages=false,bookmarksopen=true,bookmarksopenlevel=3, hyperfootnotes=false]{hyperref}

% Paquete para el manejo de tablas
\usepackage{supertabular}

% Paquetes para simbolos
\usepackage{mathcomp}
\usepackage{latexsym}
\usepackage{pifont}
\usepackage{amsfonts}
\usepackage{amssymb}
\usepackage{wasysym}
\usepackage{colortbl}
\usepackage{multicol} 
\usepackage{booktabs}

% Manejo de imagenes PDF y EPS
\newif\ifpdf
\ifx\pdfoutput\undefined
\pdffalse % we are not running PDFLaTeX
\else
\pdfoutput=1 % we are running PDFLaTeX
\pdftrue
\fi

\ifpdf
\usepackage[pdftex]{graphicx}
\else
\usepackage{graphicx}
\fi

\ifpdf
\DeclareGraphicsExtensions{.pdf, .jpg, .tif}
\else
\DeclareGraphicsExtensions{.eps, .jpg}
\fi

% Sangria de comienzo de parrafo
\setlength{\parindent}{0cm}
% Espacio vertical entre dos parrfos
\setlength{\parskip}{0.3cm}

% Definir el nombre de listas
\addto\captionsspanish{%
  \def\prefacename{Prefacio}%
  \def\refname{Referencias}%
  \def\abstractname{RESUMEN}%
  \def\bibname{BIBLIOGRAFIA}%
  \def\chaptername{Cap\'{\i}tulo}%
  \def\appendixname{ANEXO}%
  \def\listfigurename{LISTA DE FIGURAS}%
  \def\listtablename{LISTA DE CUADROS}%
  \def\indexname{\'Indice alfab\'etico}%
  \def\figurename{Figura}%
  \def\tablename{Cuadro}%
  \def\partname{Parte}%
  \def\enclname{Adjunto}%
  \def\ccname{Copia a}%
  \def\headtoname{A}%
  \def\pagename{P\'agina}%
  \def\seename{v\'ease}%
  \def\alsoname{v\'ease tambi\'en}%
  \def\proofname{Demostraci\'on}%
  \def\glossaryname{GLOSARIO}}
  \addto\captionsspanish{\def\contentsname{CONTENIDO}}

%para rotar tablas  
\usepackage{rotating}

%partir palabras
\hyphenation{di-fe-ren-cia pro-pues-to pro-ble-mas po-pu-la-res cons-truccio-nes ne-ce-sa-ria-men-te}

% Inico del documento
\begin{document}
\pagestyle{empty}% Sin número de página
%%% PORTADA


\thispagestyle{empty}

\begin{center}{\centering \large FP-FLOCK: UN ALGORITMO PARA EL DESCUBRIMIENTO DE PATRONES DE AGRUPACIÓN DE OBJETOS MÓVILES EN BASES DE
DATOS ESPACIO TEMPORALES}
\end{center}

\vspace{3cm}

\begin{center}{\large OMAR ERNESTO CABRERA ROSERO}
\end{center}

\vspace{1.5cm}

\begin{figure}[!h]
\begin{center}
\includegraphics[scale=1]{pictures/udenar.eps}%logo of your university
\end{center}
\end{figure}

\vspace{1.3cm}

\begin{center}{\large UNIVERSIDAD DE NARIÑO\\
FACULTAD DE INGENIERÍA\\
DEPARTAMENTO DE SISTEMAS\\
PROGRAMA DE INGENIERÍA DE SISTEMAS\\
SAN JUAN DE PASTO\\
2013}
\end{center}



\pagebreak

\newpage

% CONTRAPORTADA

\thispagestyle{empty}

\begin{center}
FP-FLOCK: UN ALGORITMO PARA EL DESCUBRIMIENTO DE PATRONES DE AGRUPACIÓN DE OBJETOS MÓVILES EN BASES DE
DATOS ESPACIO TEMPORALES
\end{center}

\vspace{3cm}

\begin{center}
OMAR ERNESTO CABRERA ROSERO
\end{center}

\vspace{1cm}

\begin{center}
ANTEPROYECTO DE TRABAJO DE GRADO PRESENTADO COMO REQUISITO PARA OPTAR EL TÍTULO DE  
{INGENIERO DE SISTEMAS}

\end{center}

\vspace{1cm}

\begin{center}
DIRECTOR:  ANDRES OSWALDO CALDERON ROMERO, Msc. 
\end{center}

\vspace{3cm}

\begin{center}
UNIVERSIDAD DE NARIÑO\\
DEPARTAMENTO DE SISTEMAS\\
FACULTAD DE INGENIERÍA\\
PROGRAMA DE INGENIERÍA DE SISTEMAS\\
SAN JUAN DE PASTO\\
2013
\end{center}

\pagebreak

\input{licencia.tex}
%\input{aceptacion.tex}
\tableofcontents
\input{introduccion.tex}
\input{tema.tex}
\input{descripcion.tex}
\input{objetivos.tex}
\input{justificacion.tex}
\input{marcoteorico.tex}
\input{antecedentes.tex}
\input{metodologia.tex}
\input{resultados.tex}
\input{recursos.tex}
\input{cronograma.tex}

\bibliography{biblio}
\bibliographystyle{plain}
\end{document}

